\section{Introduction}
\label{sec:introduction}

Pointer analysis underpins almost all forms of other static analyses. Some representative
applications include
program understanding, program verification, bug detection,
 security analysis,
 compiler optimization, and symbolic execution. 
 For object-oriented programs, context-sensitive pointer analyses
are the most common class of precise pointer analyses~\cite{tan2017efficient,lhotak2003scaling,smaragdakis2011pick,jeon2018precise}.
 Broadly speaking, 
there are two representative abstractions
for context-sensitivity:
(1) \emph{$k$-callsite-sensitivity}
\cite{shivers1991control}, which
distinguishes the contexts of a method by its
$k$-most-recent callsites, and (2)
\emph{$k$-object-sensitivity} \cite{milanova2005parameterized,milanova2002parameterized},
which 
distinguishes the contexts of a method
by its receiver's $k$-most-recent
allocation sites. In the past two decades, both abstractions  have been widely used in
whole-program pointer analyses \cite{bravenboer2009strictly,WALA2022,naik2006effective}, with object-sensitivity
being often preferred over callsite-sensitivity  in terms of the precision/efficiency tradeoff obtained. However, a recent study \cite{JeonPOPL22}
suggests that the converse can be true when the traditional $k$-limiting approach
is relaxed.
In the case of demand-driven pointer analyses, however, callsite-sensitivity
has been exclusively used
\cite{sridharan2006refinement,shang2012demand,yan2011demand} 
without $k$-limiting since on-demand points-to queries
can be answered subject to a per-query time budget.

\begin{comment}
program understanding \cite{sridharan2007thin}, bug detection \cite{naik2006effective, Liu2018PLDI, Fan2019ICSE},
 security analysis \cite{flowdroid2014,Fan2017ISSTA, PTaint2017OOPSLA},
 compiler optimization \cite{da2006probabilistic, sui2013query}, and symbolic execution~\cite{Trabish2018ICSE, Kapus2019FSE, Trabish2020FSE}.
\end{comment}

In this paper, we introduce the first CFL-reachability formulation of callsite-sensitive pointer analysis for object-oriented
languages.
Here, CFL stands for context-free language.
We restrict ourselves to Java, since
the underlying basic principle  applies also to other object-oriented  
languages.

Traditionally, $k$-callsite-sensitive pointer analysis (abbreviated to
 $\kcs{k}$ (Control-Flow Analysis) \cite{shivers1991control}) for Java
is either inclusion-based \cite{andersen1994program} or  founded on CFL reachability~\cite{reps1998program}. 

On one hand, Andersen-style 
inclusion-based formulation for   \kcs{k} \cite{kastrinis2013hybrid, thiessen2017context} has been adopted in several pointer analysis frameworks for Java
\cite{naik2006effective,vallee2010soot,WALA2022,bravenboer2009strictly, he_et_al:LIPIcs.ECOOP.2022.30}. 
%, e.g.,  \jchord~\cite{naik2006effective},  \soot~\cite{vallee2010soot}, \wala~\cite{WALA2022},   \doop~\cite{bravenboer2009strictly}, and\qilin~\cite{he_et_al:LIPIcs.ECOOP.2022.30}. 
 Given a  program, its statements are modeled as  points-to set constraints,  its methods' calling contexts (abstracted by their
 last $k$
 callsites)  are tracked by parameterizing these constraints with context abstractions, and its call graph 
 is often constructed on the fly in order to achieve  the best precision and efficiency possible~\cite{grove2001framework, ryder2003dimensions, lhotak2003scaling, lhotak2008evaluating, smaragdakis2011pick}.

On the other hand, the
CFL-reachability formulation for \kcs{k}~\cite{sridharan2006refinement} has also been used extensively in  understanding and developing  a wide range of pointer analysis algorithms, such as demand-driven pointer/alias analysis~\cite{sridharan2005demand, sridharan2006refinement, zheng2008demand, yan2011demand, shang2012demand}, context transformations~\cite{thiessen2017context}, 
specification inference \cite{Bastani15PLDI},
library-code summarization \cite{shang2012demand, tang2015summary},
  information flow analysis \cite{li2020fast,milanova2020flowcfl},
and selective context-sensitivity~\cite{lu2021selective,li2018precision}. 
Given a program,  its points-to information is computed by solving a graph reachability problem over
 a so-called \textit{pointer assignment graph} (PAG)~\cite{lhotak2003scaling}. 
Such a CFL-reachability formulation consists of reasoning about the intersection of two CFLs, \manuCapLFC, 
where \manuLF describes field accesses as balanced parentheses and \manuLC enforces 
  callsite-sensitivity by matching method calls and returns as also balanced parentheses~\cite{sridharan2006refinement}. However,
 a separate algorithm, which is formulated outside \manuLFC, is used for call graph construction (as
 described in \Cref{sec:backgroundAndMotivation}).

Compared with Andersen-style inclusion-based formulation, 
this \manuLFC-based CFL-reachability formulation for
specifying \kcs{k}
suffers from two major limitations due to the lack of  a built-in call graph construction mechanism.
First, \kcs{k} may lose precision even if it uses the most precise call graph
for a program (built in advance or on the fly).
%using, for example, CHA~\cite{dean1995optimization}, RTA~\cite{bacon1996fast} and VTA~\cite{sundaresan2000practical}).
Second, any  analysis that reasons about  CFL-reachability in terms of \manuLFC will fail
to make a connection with
the value-flow paths traversed by a separate call graph construction
algorithm used, and consequently, may
make optimization decisions 
that reduce  the precision of \kcs{k} (among others).


\begin{comment}
In \cite{sridharan2005demand, sridharan2006refinement}, where the callsite-based CFL reachability formulation was first proposed for handling demand-driven pointer analysis, the authors has to resort to an additional algorithm for on-fly callgraph construction. However, in a considerable part of work, algorithm-assisted on-fly construction approaches are insufficient to complete their tasks. For instance, \selectx , a recent effort for \kcs{k} \cite{lu2021selective}, performs its context-sensitivity selection analysis in a pre-analysis stage and 
cannot rely on an expensive additional algorithm for on-fly dispatching. Consequently, \selectx suffers from a few precision losses.
Thus, a language-built-in formulation for on-fly callgraph construction is urgently required.
\end{comment}
% especially in those in recent years, graph reachability plays an increasingly important role, which cannot be replaced by only specific algorithms.

% In traditional callsite-based CFL formulation~\cite{sridharan2006refinement, yan2011demand}, $L_F$ only describes value flows, leaving call-graph building unmentioned, 
% In practice, the call-graph can be created in advance by some pre-analysis (such as IFDS~\cite{}), in such way $L_F$ can describe all the pointer relationships in the program (by regarding method calls and returns as assignments).
% But there is a loss of precision in doing so, and the precision loss depends on the difference between the call-graph given by the pre-analysis and the actual call-graph (the one which is on-fly built).
%Of course, if the call-graph given by the pre-analysis is consistent with the actual call-graph, there is no loss of precision then, 
%but this also means that the pre-analysis should have done the work of $L_F$ and make $L_F$ meaningless.

% Alternatively, the call graph can usually be constructed on the fly, 
% but its construction process is usually described by a specific algorithm (for example, demand-driven pointer analysis ~\cite{sridharan2005demand, sridharan2006refinement}), rather than a language like value-flow.

In this paper, as  \emph{the primary contribution of this research}, we overcome these two limitations by introducing a  CFL-reachability formulation of \kcs{k}  for Java with  built-in on-the-fly call graph construction (as part of the
same CFL-reachability formulation).
%However, how to design a CFL-reachability formulation for callsite-based context-sensitive pointer analysis with on-fly callgraph construction being supported remains to be unknown.
% Using graph reachability to explicitly describe on-fly call-graph building 
%It was rarely discussed in past literals, due to its extraordinary complexity. 
%
% We are not aware of any earlier attempt on studying this
% problem in the literature.
\hl{Adapting the previous formulation used in object-sensitive pointer analysis \cite{lu2019precision, lu2021eagle} to our approach poses significant challenges and complexities, as discussed in Section \ref{paragraph:chl2} and further explored in Section \ref{sec:relatedwork}. Furthermore, an earlier attempt to address the same problem, presented in Sridharan's PhD thesis \cite{sridharan2007refinement}, falls short in providing a comprehensive solution, as will be discussed in detail in Section \ref{sec:relatedwork}.}
\begin{comment}
First, the CFL-reachability formulation does not have an equivalent set-constraint formulation as it is in the form of an intersection of several CFLs which has been proven to be undecidable \cite{reps2000undecidability}. Thus, attempting to obtain the CFL-reachability formulation from their corresponding set constraints by exploiting Melski-Reps reduction \cite{melski2000interconvertibility} is infeasible. Second, even we could use Melski-Reps reduction \cite{melski2000interconvertibility} to obtain a CFL-reachability formulation to track context-insensitive value flow with on-fly callgraph construction included, needless to say how complicated its form is, how to adapt it to support fully callsite-sensitivity is still a great challenge.
\end{comment}
% It is true that in many tasks that only calculate value-flow, the specific algorithm description is sufficient to complete these tasks, thus avoiding this problem.
% However, in a considerable part of the work, especially in those in recent years, graph reachability plays an increasingly important role, which cannot be replaced by only specific algorithms.
% For example
Our formulation consists of reasoning about the intersection of three  CFLs, \capLFCR, 
over a new PAG representation for a Java program,
where \LF specifies not only field accesses
as in \manuLF but also  dynamic method dispatch, \LC  enforces  callsite-sensitivity
exactly as before \cite{sridharan2006refinement}, and  \LR supports parameter passing in the presence of built-in
on-the-fly call graph construction.
%ensures that the context used for parameter passing at a callsite remains unchanged  after a  CFL-reachability traversal has been performed for dispatching
%a virtual method invoked at the callsite.  
\hl{Theoretically, we present a novel insight by demonstrating for the first time that callsite-sensitive pointer analysis can be formulated as a particular type of context-sensitive language. Specifically, we express it as the intersection of multiple CFLs. It is important to note that not all context-sensitive languages can be expressed in this manner\cite{liu1973infinite, Latta1993phdthesis}, highlighting the uniqueness of our approach.}
We will discuss several challenges faced in designing \LFCR
and provide some insights for understanding our  formulation. Note that the
 Melski-Reps reduction \cite{melski2000interconvertibility} cannot be used to convert Andersen-style
 inclusion-based formulation for \kcs{k} into \LFCR  since  \LFCR is the intersection of three different
 CFLs rather than just one single CFL.
 
\LFCR can be applied to all the applications that benefit from
\manuLFC.
By formulating \kcs{k} 
\hl{(with built-in on-the-fly callgraph construction)} % completely 
in terms of CFL-reachability,  
\LFCR can be potentially more useful than \manuLFC
for several recently-studied CFL-reachability-based
analyses:
specification inference \cite{Bastani15PLDI},
library-code summarization \cite{shang2012demand, tang2015summary},
  information flow analysis \cite{li2020fast, milanova2020flowcfl}, and
 selective context-sensitivity \cite{lu2021selective,li2018precision}. 
  To demonstrate
its utility,  as \emph{the secondary contribution of this research}, we introduce the first \LFCR-enabled precision-preserving pre-analysis
 for accelerating \kcs{k} for Java with selective context-sensitivity,
 which also serves to validate 
 the correctness of \LFCR .
 In contrast,
 a recently proposed pre-analysis \cite{lu2021selective} will always lose precision as it
 is developed 
 based on \manuLFC~\cite{sridharan2006refinement}. 
 %We will explain why a precision-preserving pre-analysis can be developed from \LFCR but not from \LFC.

\begin{comment}
Specifically, we have realized \tool, the first precision-preserving acceleration approach for \kcs{k}, which in
turn proved the correctness of our CFL-reachability formulation. During our design, we have applied a similar
over-approximation approach on CFLs as \selectx \cite{lu2021selective} to first obtain a regular language $L_R$. 
Next, we perform context-sensitivity selection by reasoning about object reachability guided by $L_R$ and $L_C$. As a result, \tool is simple, lightweight, and very efficient due to its selection algorithm is actually a taint analysis in essence.
\end{comment}
% \tool applies the exactly same over-approximation approach on CFLs as \selectx \cite{lu2021selective} but achieves a dramatically poor efficiency in its context selection. We thus proposed two optimization approaches: (1) \tooll improves the selection efficiency at the cost of a negligible precision loss in main analysis by pre-excluding a ton of redundant virtual call dispatching value flows for monomorphic calls precomputed by \spark \cite{lhotak2003scaling} (a context-insensitive Andersen's analysis \cite{andersen1994program}). (2) \toolp improves the selection efficiency while still preserving precision of main analysis by further over-approximating \selectx's context selection algorithm into a taint analysis. 



% In general, \toolp could accelerate \kcs{k} \OVERALLEagleCSPSPEEDUP and \tooll achieves a slightly better speedups (\OVERALLEagleCSLSPEEDUP) at the cost of a negligible precision loss. The average context selection time has also been significantly reduced from \toolPreTime seconds (for \tool) to \toolpPreTime seconds by \toolp and \toollPreTime seconds by \tooll respectively. 

% In this work, we mainly discussed the necessity, complexity, and feasibility of using graph reachability to explicitly describe on-fly call-graph building,
% and finally gave a feasible CFLs for a field-sensitive and callsite-sensitive pointer analysis including on-fly call-graph building.
% We also present an application of it -- the precision-preserving acceleration of \kcs{k}, the experimental results of which prove its correctness and effectiveness.

In summary, this paper makes the following two  contributions:
\begin{itemize}
    \item Our primary contribution is to introduce a CFL-reachability formulation \LFCR of
    \kcs{k}  for object-oriented languages 
   with on-the-fly call graph construction being built into the formulation itself (rather than done by
   a separate call graph construction algorithm), demonstrating for the first time that callsite-sensitive pointer analysis represents a special kind of context-sensitive language that can be expressed by the intersections of several CFLs.
    
    \item Our secondary contribution is to introduce an \LFCR-enabled precision-preserving pre-analysis for 
    accelerating \kcs{k} for object-oriented languages with selective context-sensitivity. 
    Compared with two state-of-the-art pre-analyses  \cite{lu2021selective,li2018precision}, our pre-analysis 
    enables
better efficiency-precision trade-offs to be made in several application scenarios outlined.
\begin{comment}    
    Our evaluation, conducted
in   \qilin~\cite{he_et_al:LIPIcs.ECOOP.2022.30} using 
a set of 13 popular Java benchmarks/applications, shows that our pre-analysis
enables  
\kcs{k} to achieve an average speedup of 
\OVERALLTOOLSPEEDUP while incurring  %negligible 
small pre-analysis overheads (\toolPreTime seconds on average).
%\tool is only \toolPreTime seconds.
\end{comment}
\end{itemize}

The rest of this paper is organized as follows.
\Cref{sec:backgroundAndMotivation} provides some background knowledge and motivates the
development of \LFCR by highlighting several challenges faced in its design. \Cref{sec:CFL} introduces 
\LFCR by explaining  how we address these challenges and providing some
insights in understanding its design.   \Cref{sec:EagleCS}  introduces and evaluates our \LFCR-enabled  pre-analysis for accelerating
\kcs{k}. \Cref{sec:relatedwork}
 discusses the related work.  \Cref{sec:conclusion} concludes the paper.